\documentclass{beamer}
\usepackage[UTF8]{ctex}
\usepackage{graphicx} 
\usepackage{float} 
\usepackage{subfigure}


\usetheme{Madrid}
\usecolortheme{default}

%------------------------------------------------------------
%This block of code defines the information to appear in the
%Title page
\title %optional
{基于同构测试的图神经网络简介}

\author % (optional)
{吴钰同}

\date % (optional)
{\today}

%End of title page configuration block
%------------------------------------------------------------



%------------------------------------------------------------
%The next block of commands puts the table of contents at the 
%beginning of each section and highlights the current section:

\AtBeginSection[]
{
  \begin{frame}
    \frametitle{Table of Contents}
    \tableofcontents[currentsection]
  \end{frame}
}
%------------------------------------------------------------


\begin{document}

%The next statement creates the title page.
\frame{\titlepage}


%---------------------------------------------------------
%This block of code is for the table of contents after
%the title page
\begin{frame}
\frametitle{目录}
\tableofcontents
\end{frame}
%---------------------------------------------------------


\section{Weisfeiler-Lehman Test}

%---------------------------------------------------------
%Changing visivility of the text
\begin{frame}

  \frametitle{图同构}

  \begin{block}{定义}
    对于两个无向图 $G$ 和 $H$,若存在它们顶点之间的双射 $f:V(G) \rightarrow V(H)$,
    使得 $G$ 中的顶点 $u$ 和 $v$ 相邻当且仅当 $H$ 中的顶点 $f(u)$ 和 $f(v)$ 相邻,
    则称 $G$ 和 $H$ 同构。
  \end{block}
  \centering
  \includegraphics[scale=0.1]{figs/Graph_isomorphism_a.png}
  \includegraphics[scale=0.2]{figs/Graph_isomorphism_b.png}

\end{frame}

%---------------------------------------------------------


%---------------------------------------------------------
%Example of the \pause command
\begin{frame}
In this slide \pause

the text will be partially visible \pause

And finally everything will be there
\end{frame}
%---------------------------------------------------------

\section{Second section}

%---------------------------------------------------------
%Highlighting text
\begin{frame}
\frametitle{Sample frame title}

In this slide, some important text will be
\alert{highlighted} because it's important.
Please, don't abuse it.

\begin{block}{Remark}
Sample text
\end{block}

\begin{alertblock}{Important theorem}
Sample text in red box
\end{alertblock}

\begin{examples}
Sample text in green box. The title of the block is ``Examples".
\end{examples}
\end{frame}
%---------------------------------------------------------


%---------------------------------------------------------
%Two columns
\begin{frame}
\frametitle{Two-column slide}

\begin{columns}

\column{0.5\textwidth}
This is a text in first column.
$$E=mc^2$$
\begin{itemize}
\item First item
\item Second item
\end{itemize}

\column{0.5\textwidth}
This text will be in the second column
and on a second tought this is a nice looking
layout in some cases.
\end{columns}
\end{frame}
%---------------------------------------------------------


\end{document}